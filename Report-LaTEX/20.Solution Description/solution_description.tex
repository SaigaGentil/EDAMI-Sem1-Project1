\chapter{Solution Description}
\label{chap:solution-description}

This project proposes implementing the Charm algorithm to generate non-redundant association rules based on closed frequent itemsets. Charm (Closed itemset Miner) is a depth-first search algorithm specifically designed to discover closed frequent itemsets efficiently. By leveraging Charm, we can obtain a compact and non-redundant set of itemsets, ensuring the generated association rules are informative and meaningful.

The solution is implemented in Python and involves several steps. First, the Charm algorithm is used to mine closed frequent itemsets from the dataset. The algorithm explores the lattice of itemsets, pruning unnecessary search branches and avoiding redundant itemset generation. This step guarantees that only closed frequent itemsets, which have no supersets with the same support, are considered.

Once the closed frequent itemsets are obtained, association rules can be generated. A pruning technique is applied to eliminate rules that are subsumed by other rules with higher confidence. This pruning step ensures that only the most interesting and distinct rules are retained, further reducing redundancy.

The solution will be experimented on the Traffic Accidents data set (http://fimi.uantwerpen.be/data/accidents.pdf) <TODO: Add proper citation with bibtex later> to evaluate its performance. The results will be compared with those obtained by other algorithms like Apriori to determine the effectiveness of the proposed solution.

Below is pseudocode for the Charm algorithm:
- Create a function to extract unique items from the data. The function should take in a dataset as a parameter and return a list of unique items in the dataset.
- Create a function to calculate the support of an itemset in the data set. The function should take in a dataset and an itemset as parameters and return the support of the itemset in the dataset.
- Create a function, Charm, that takes in a dataset and a minimum support threshold as parameters. The function should, somehow (Do be defined in the final document), use the two functions defined above and return a list of closed frequent itemsets.
- Once the closed frequent itemsets are obtained, association rules can be generated. A pruning technique is applied to eliminate rules that are subsumed by other rules with higher confidence. This pruning step ensures that only the most interesting and distinct rules are retained, further reducing redundancy.

<TODO: Research and read literature to expand more on the above proposed solution and how to implement>